%!TEX root = ../main.tex
%%%%%%%%%%%%%%%%%%%%%%%%%%%%%%%%%%%%%%%%%
%
%LEZIONE 22/02/2016 - PRIMA SETTIMANA (1)
%
%%%%%%%%%%%%%%%%%%%%%%%%%%%%%%%%%%%%%%%%%
\chapter{Topologia Generale}

La \emph{topologia} studia gli aspetti geometrici più generali, senza introdurre né strutture algebriche, né metriche, né coordinate.
%%%%%%%%%%%%%%%
%SPAZI METRICI%
%%%%%%%%%%%%%%%
\section{Spazi metrici}

\begin{defn}{Spazio metrico}{spazioMetrico}\index{Spazio!metrico}
    Sia $X\neq\emptyset$ un insieme.
    Diremo che $X$ è uno \emph{spazio metrico} se è definita un'applicazione
    \[
        d\colon X\times X\to\R,(x,y)\mapsto d(x,y),
    \]
    detta \emph{distanza}, che rispetti le seguenti proprietà:
    \begin{description}
        \item[Positività]$d(x,y)\ge 0,\,\fa x,y\in X$ e $d(x,y)=0\iff x=y$;
        \item[Simmetria]$d(x,y)=d(y,x),\,\fa x,y\in X$;
        \item[Triangolare]$d(x,y)\le d(x,z)+d(z,y),\,\fa x,y,z\in X$.
    \end{description}
\end{defn}

\begin{notz}\index{Funzione}
    Da questo momento con \emph{funzione} denoteremo un'applicazione che ha $\R$ come codominio.
\end{notz}

\begin{exe}
    $\R$ è uno spazio metrico, ponendo
    \[
        d(x,y)=\abs{y-x}.
    \]
\end{exe}

\begin{sol}
    \'E sufficiente verificare che le proprietà della definizione \ref{df:spazioMetrico} siano verificate.
    La positività segue banalmente dalla definizione di valore assoluto.
    Per la simmetria osserviamo che
    \[
        d(x,y)=\abs{y-x}=\abs{-(x-y)}=\abs{x-y}=d(y,x).
    \]
    Infine, dal momento che la disuguaglianza triangolare è valida per il valore assoluto, segue
    \[
        d(x,y)=\abs{y-x}=\abs{y-z+z-x}\le\abs{y-z}+\abs{z-x}=d(z,y)+d(x,z).
    \]
\end{sol}

\begin{ese}
    $\R^n$ è uno spazio metrico, ponendo
    \[
        d(\vec{x},\vec{y})=\norma{\vec{y}-\vec{x}}.
    \]
\end{ese}

\begin{ese}
    Sia $X\neq\emptyset$ un insieme qualsiasi, $X$ è uno spazio metrico ponendo
    \[
        d(x,y)= \begin{cases}
            0 & x=y     \\
            1 & x\neq y
        \end{cases},
    \]
    tale applicazione si definisce \emph{distanza discreta}.
\end{ese}

\begin{prop}{Sottoinsieme di uno spazio metrico è uno spazio metrico}{}
    Sia $(X,d)$ uno spazio metrico e sia $Y\subset X$, allora:
    $(Y,\left. d\right|_Y)$ è uno spazio metrico, con
    \[
        \left. d\right|_Y\colon Y\times Y\to\R,(y',y'')\mapsto d(y',y'').
    \]
\end{prop}

\begin{proof}
    $(X,d)$ è uno spazio metrico, quindi per definizione $d$ costituisce una distanza su $X$. Consideriamo quindi la restrizione $\left. d\right|_Y$ di $d$ su $Y$, tale applicazione soddisfa necessariamente le proprietà della distanza in quanto è definita a partire da $d$. Pertanto $(Y,\left. d\right|_Y)$ è uno spazio metrico.
\end{proof}

\begin{notz}
    $\left. d\right|_Y$ di definisce \emph{distanza indotta}.
\end{notz}

\begin{defn}{Sottospazio metrico}{sottospazioMetrico}\index{Sottospazio!metrico}
    Sia $(X,d)$ uno spazio metrico e sia $Y\subset X$.
    $(Y,\left. d\right|_Y)$ si definisce \emph{sottospazio metrico} di $(X,d)$.
\end{defn}

\begin{defn}{Applicazione continua tra spazi metrici}{applContinuaSpMet}\index{Applicazione continua!tra spazi metrici}
    Siano $(X,d_X)$ e $(Y,d_Y)$ due spazi metrici. Un'applicazione $f\colon (X,d_X)\to (Y,d_Y)$ si dice \emph{continua} se, comunque preso $x_0\in X$
    \[
        \fa\e >0\,\ex\d=\d(\e)>0\text{ tale che }d_X(x,x_0)<\d\implies d_Y\big(f(x),f(x_0)\big)<\e.
    \]
\end{defn}

\begin{exe}[per casa]
    Si dimostri che ogni applicazione costante tra spazi metrici è continua.
\end{exe}

\begin{sol}
    Siano $(X,d_X)$ e $(Y,d_Y)$ spazi metrici, definiamo quindi una generica applicazione costante
    \[
        f\colon X\to Y,x\mapsto y_0\in Y,
    \]
    con $y_0$ fissato.
    Sia ora $x_0\in X$ e fissiamo $\e >0$, osserviamo quindi che, preso un qualsiasi $x\in X$, avremo
    \[
        d_Y\big(f(x),f(x_0)\big)=d_Y(y_0,y_0)=0<\e,
    \]
    indipendentemente da $d_X(x,x_0)$.
    Dalla definizione \ref{df:applContinuaSpMet} segue quindi che $f$ è continua.
\end{sol}

\begin{exe}[per casa]
    Sia $(X,d)$ uno spazio metrico, si dimostri che $id_X\colon X\to X,x\mapsto x$ è continua.
\end{exe}

\begin{sol}
    Sia $x_0\in X$ e fissiamo $\e >0$, poniamo quindi $\d>0$ tale che $\d < \e$, da cui
    \[
        d(x,x_0)<\d\implies d\big(id(x),id(x_0)\big)=d(x,x_0)<\d<\e,
    \]
    ovvero $f$ è continua.
\end{sol}
%%%%%%%%%%%%%%%%
%INSIEMI APERTI%
%%%%%%%%%%%%%%%%
\section{Insiemi aperti}

\begin{defn}{Disco aperto}{discoAperto}\index{Disco aperto}
    Sia $(X,d)$ uno spazio metrico. Sia $x\in X$ e sia $r>0$.
    Si definisce \emph{disco aperto} di centro $x$ e raggio $r$, il sottoinsieme di $X$ di tutti i punti la cui distanza da $x$ è inferiore ad $r$:
    \[
        D_x(r)=\Set{y\in X | d(y,x)<r}.
    \]
\end{defn}

\begin{ese}
    Su $\big(\R,\abs{.}\big)$ i dischi aperti di centro $x$ e raggio $r$ sono gli intervalli aperti $(x-r,x+r)$.
\end{ese}

\begin{ese}
    Su $\big(\R^n,\norma{.}\big)$ i dischi aperti sono palle aperte.
\end{ese}

\begin{defn}{Insieme aperto}{aperto}\index{Insieme!aperto}
    Sia $(X,d)$ uno spazio metrico e sia $A\subset X$.
    $A$ è un insieme aperto se può essere scritto come unione di dischi aperti.
\end{defn}

\begin{oss}
    Equivalentemente possiamo definire $A$ aperto se
    \[
        \fa x\in A\,\ex r>0:D_x(r)\subset A,
    \]
    infatti basterà mostrare che
    \[
        A=\bigcup_{x\in A}D_x(r).
    \]
    \graffito{$\subseteq$}Sia $z\in A$, avremo quindi $z\in D_z(r)\implies z\in\bigcup_{x\in A}D_x(r)$;

    \graffito{$\supseteq$}Sia $z\in\bigcup_{x\in A}D_x(r)$, per cui $z\in D_z(r)$, ma per ipotesi $D_z(r)\subset A$, ovvero $z\in A$.
\end{oss}

\begin{ese}
    Sono aperti di $\big(\R,\abs{.}\big)$ i seguenti:
    \begin{itemize}
        \item $\displaystyle{\R=\bigcup_{n\in \Z} D_n(1)}$, oppure $\displaystyle{\R=\bigcup_{n\in \Z} D_0(n)}$;
        \item $\emptyset$;
        \item $(a,b)=D_c(\frac{b-a}{2})$, dove $c=\frac{a+b}{2}$ è il punto medio fra $a,b$;
        \item $(a,+\infty)$;
        \item $(-\infty,b)$.
    \end{itemize}
\end{ese}

\begin{ese}
    Non sono aperti di $\big(\R,\abs{.}\big)$ i seguenti:
    \begin{itemize}
        \item Insiemi finiti;
        \item $\Z$;
        \item $\Q$;
        \item $\R\setminus\Q$.
    \end{itemize}
\end{ese}

\begin{exe}[per casa]
    Su $\big(\R,\abs{.}\big)$ si dimostri che un intervallo chiuso $[a,b]$ non è un insieme aperto.
\end{exe}

\begin{sol}
    Se per assurdo $[a,b]$ fosse aperto, avremmo che
    \[
        \fa x\in [a,b]\,\ex r>0:D_x(r)\subset [a,b],
    \]
    in particolare $\ex r>0:D_a(r)\subset [a,b]$, dove
    \[
        D_a(r)=\Set{x\in\R : \abs{a-x}<r}.
    \]
    Sia ora $0<r'<r$, segue banalmente che $a-r'\in D_a(r)$, ma ciò è assurdo, in quanto $a-r'\notin [a,b]$.
    Quindi $[a,b]$ non è un aperto di $\R$.
\end{sol}

\begin{exe}[per casa]
    Su $\big(\R,\abs{.}\big)$ si dimostri che un intervallo semichiuso $[a,b)$ non è un insieme aperto.
\end{exe}

\begin{sol}
    Analoga alla precedente.
\end{sol}

\begin{teor}{Continuità definita per aperti}{continuitàAperti}\index{Teorema!di continuità definita per aperti}
    Siano $(X,d_X)$ e $(Y,d_Y)$ spazi metrici e sia $f\colon X\to Y$, allora:
    \[
        f\text{ è continua}\iff f^{-1}(A)\text{ è aperto },\,\fa A\subset Y\text{ aperto.}
    \]
\end{teor}

\begin{proof}
    \graffito{$\Rightarrow)$}Sia $\emptyset\neq A\subset Y$ aperto e sia $x\in f^{-1}(A)\subset X$, vogliamo dimostrare che $f^{-1}(A)$ è aperto.
    Dobbiamo quindi mostrare che
    \[
        \ex r>0:D_x(r)\subset f^{-1}(A).
    \]
    Ora $A$ aperto implica
    \[
        \ex\e>0:D_{f(x)}(\e)\subset A,\,\fa f(x)\in A,
    \]
    inoltre $f$ continua, per cui
    \[
        \ex\d=\d(\e):f\big(D_x(\d)\big)\subset D_{f(x)}(\e),
    \]
    in particolare
    \[
        D_{f(x)}(\e)\subset A\implies f\big(D_x(\d)\big)\subset A,
    \]
    ovvero
    \[
        D_x(\d)\subset f^{-1}(A).
    \]
    \graffito{$\Leftarrow)$}La dimostrazione è analoga.
\end{proof}

\begin{oss}
    La nozione di applicazione continua su uno spazio metrico qualsiasi, non dipende quindi dalla distanza, ma solamente dalla nozione di insieme aperto.
\end{oss}
%%%%%%%%%%%%%%%%%%%%%%%%%%%%%%%%%%%%%%%%%
%
%LEZIONE 24/02/2016 - PRIMA SETTIMANA (2)
%
%%%%%%%%%%%%%%%%%%%%%%%%%%%%%%%%%%%%%%%%%
%%%%%%%%%%%%%%%%%%
%SPAZI TOPOLOGICI%
%%%%%%%%%%%%%%%%%%
\section{Spazi topologici}

\begin{defn}{Topologia su un insieme}{topologiaSuInsieme}\index{Topologia}
    Sia \(X\neq\emptyset\) un insieme.
    Una famiglia \(\Tau\) di sottoinsiemi di \(X\), detti \emph{insiemi aperti}, si definisce una \emph{topologia} su \(X\) se soddisfa le seguenti proprietà:
    \begin{enumerate}
        \item \(\emptyset\) e \(X\) appartengo a \(\Tau\);
        \item l'unione qualsiasi di aperti è un aperto;
        \item l'intersezione finita di aperti è un aperto.
    \end{enumerate}
\end{defn}

\begin{notz}
    Se \(A\in \Tau\), si dice che \(A\) è un aperto di \(X\).
\end{notz}

\begin{defn}{Spazio topologico}{spazioTopologico}\index{Spazio!topologico}
    Sia \(X\neq\emptyset\) un insieme e sia \(\Tau\) una topologia su \(X\).
    Si definisce \emph{spazio topologico} la coppia \(X,\Tau\).
\end{defn}

\begin{notz}
    Gli elementi \(x\in X\) si definiscono \emph{punti} di \(X\).
\end{notz}

Le definizioni che seguono mostrano alcuni esempi di spazi topologici.

\begin{defn}{Topologia indotta dalla metrica}{topologiaIndottaMetrica}\index{Topologia!indotta dalla metrica}
    Sia \((X,d)\) uno spazio metrico.
    Gli insiemi aperti della definizione \ref{df:aperto} inducono una topologia su \(X\), detta \emph{topologia indotta dalla metrica}.
\end{defn}

\begin{oss}
    Mostriamo che tale topologia verifica le proprietà:
    \begin{enumerate}
        \item Banalmente
              \[
                  \emptyset =\bigcup_\emptyset D_x(r),
              \]
              e
              \[
                  X=\bigcup_{x\in X}D_x(r).
              \]
        \item Per definizione \(A_i\) è aperto se e soltanto se
              \[
                  A_i=\bigcup_{\substack{x_i\in X\\r_i>0}}D_{x_i}(r_i),
              \]
              per cui
              \[
                  \begin{split}
                      \bigcup_{i\in I}A_i & =\bigcup_{i\in I}\bigcup_{\substack{x_i\in X\\r_i>0}}D_{x_i}(r_i)\\
                      & =\bigcup_{i\in I}D_{x_i}(r_i),
                  \end{split}
              \]
              ovvero
              \[
                  \bigcup_{i\in I}A_i
              \]
              è aperto.
        \item Siano \(A_1,\dots,A_n\in\Tau\), dobbiamo mostrare che
              \[
                  \bigcap_{i=1,\dots,n}A_i\in\Tau.
              \]
              Ma ciò vale se e soltanto se
              \[
                  B_1,B_2\in\Tau\implies B_1\cap B_2\in\Tau,
              \]
              in quanto ciò ci permette di procedere per induzione.
              Nel nostro caso abbiamo infatti:
              \[
                  B_1\cap B_2=\bigcup_{\substack{x\in B_1\cap B_2\\r:D_x(r)\subset B_1\cap B_2}}D_x(r).
              \]
    \end{enumerate}
\end{oss}

\begin{ese}
    La topologia naturale di \(\R^n\) è lo spazio metrico \(\big(\R^n,\norma{.}\big)\).
\end{ese}

\begin{defn}{Topologia banale}{topologiaBanale}\index{Topologia!banale}
    Sia \(X\) un insieme qualsiasi.
    \(\Tau=\Set{\emptyset,X}\) si definisce \emph{topologia banale} di \(X\).
\end{defn}

\begin{defn}{Topologia discreta}{topologiaDiscreta}\index{Topologia!discreta}
    Sia \(X\) un insieme qualsiasi.
    \(\Tau=\mathcal{P}(X)\) si definisce \emph{topologia discreta} di \(X\).
\end{defn}

\begin{exe}
    Sia \(X=\Set{a,b,c,d}\) un insieme di \(4\) elementi.
    Siano
    \begin{gather*}
        \Tau=\Set{X,\emptyset,\Set{a},\Set{a,b},\Set{a,b,c},\Set{a,b,d}}\\
        \mathrm{S}=\Set{X,\emptyset,\Set{b},\Set{b,c},\Set{a,b,c}}.
    \end{gather*}
    Stabilire se \(\Tau\) o \(\mathrm{S}\) sono due topologie diverse dello stesso insieme \(X\).
\end{exe}

\begin{sol}
    Verificando le proprietà segue banalemte che entrambe sono topologie di \(X\), inoltre
    \[
        \Set{a}\notin \mathrm{S}\text{ e }\Set{b}\notin\Tau,
    \]
    per cui non costituiscono le stesse topologie.
\end{sol}

\begin{exe}
    Rifacendosi all'esercizio precedente, stabile se
    \[
        \mathrm{R}=\Set{X,\emptyset,\Set{a},\Set{b},\Set{a,b},\Set{a,b,c},\Set{b,c,d}}
    \]
    è una topologia su \(X\).
\end{exe}

\begin{sol}
    No, in quanto
    \[
        \Set{a,b,c}\cap\Set{b,c,d}=\Set{b,c},
    \]
    ma
    \[
        \Set{b,c}\notin\mathrm{R}.
    \]
\end{sol}

\begin{defn}{Relazione di finezza}{finezza}\index{Finezza}
    Sia \(X\) un insieme su cui sono definite due topologie \(\Tau\) e \(\Tau'\).
    Diremo che \(\Tau'\) è più \emph{fine} di \(\Tau\) se ogni aperto di \(\Tau\) è anche aperto di \(\Tau'\).
\end{defn}

\begin{notz}
    Si scrive \(\Tau'>\Tau\).
\end{notz}

\begin{oss}
    Qualunque sia \(X\) avremo sempre che la topologia banale è la meno fine, mentre quella discreta è la più fine.
\end{oss}

\begin{defn}{Base}{base}\index{Base}
    Una base di uno spazio topologico \((X,\Tau)\) è una famigli di aperti \(\mathcal{B}\) tale che ogni aperto di \(X\) è unione di elementi di \(\mathcal{B}\).
\end{defn}

\begin{exe}[per casa]
    Mostrare che, equivalentemente, \(\mathcal{B}\) è una base se e soltanto se
    \begin{equation}\label{eq:base1}
        \fa A\subset X\text{ aperto e }\fa x\in A\,\ex B\in\mathcal{B}:x\in B\subseteq A.
    \end{equation}
\end{exe}

\begin{sol}
    \graffito{\(\Rightarrow)\)}Supponiamo che \(\mathcal{B}\) sia una base di \(\Tau\) e sia \(A\) un aperto di \(X\), per cui
    \[
        A=\bigcup_{i\in I}B_i,\text{ con }B_i\in\mathcal{B}.
    \]
    Sia ora \(x\in A\), quindi \(\ex i\in I\) tale che \(x\in B_i\).
    Resta da mostrare che \(B\subseteq A\), ma ciò segue immediatamente dalla scrittura di \(A\) come unione di \(B_i\).\\
    \graffito{\(\Leftarrow)\)}Supponiamo che \(\mathcal{B}\) soddisfi \eqref{eq:base1} e sia \(A\) un aperto di \(X\).
    Noi vorremmo mostrare che
    \[
        A=\bigcup_{x\in A}B_x,\text{ con }B_x\in\mathcal{B}.
    \]
    Sia quindi \(x\in A\), avremo quindi che
    \[
        \ex B_x\in\mathcal{B}:x\in B_x\subseteq A,
    \]
    ovvero
    \[
        A\subseteq\bigcup_{x\in A}B_x.
    \]
    Sia \(y\in\bigcup_{x\in A}B_x\), avremo \(\ex B_x:y\in B_x\), ma \(B_x\in \mathcal{B}\), quindi, per ipotesi, \(B\subseteq A\), ovvero
    \[
        A\supseteq\bigcup_{x\in A}B_x.
    \]
\end{sol}

\begin{ese}
    Gli intervalli limitati sono una base della topologia naturale di \(\R\).
\end{ese}

\begin{ese}
    Più in generale, i dischi di uno spazio metrico, sono una base della topologia indotta dalla metrica, ovvero
    \[
        \mathcal{B}=\Set{D_x(r) | x\in X,r>0}
    \]
    è una base di \((X,d)\).
\end{ese}

\begin{prop}{Caratterizzazione delle basi}{caratBasi}
    Sia \(\mathcal{B}\) una famiglia di sottoinsiemi di \(X\) tale che:
    \begin{enumerate}
        \item \(\displaystyle\bigcup_{b\in\mathcal{B}}B=X\), ovvero \(\mathcal{B}\) è un ricoprimento di \(X\);
        \item \(\fa A,B\in \mathcal{B}\implies A\cap B\) è unione di elementi di \(\mathcal{B}\).
    \end{enumerate}
    Allora esiste un'unica topologia \(\Tau_{\mathcal{B}}\) su \(X\) tale che \(\mathcal{B}\) è una base di \(\Tau_{\mathcal{B}}\).
\end{prop}

\begin{proof}
    Definiamo \(\Tau_{\mathcal{B}}\) come la famiglia dei sottoinsiemi di \(X\) che si possono scrivere come unione (tipicamente infinita) di elementi di \(\mathcal{B}\).
    Tale scelta definisce \(\Tau_{\mathcal{B}}\) in maniera univoca.
    Dobbiamo quindi dimostrare che \(\Tau_{\mathcal{B}}\) è una topologia su \(X\):
    \begin{itemize}
        \item \(\displaystyle X\overset{(1)}{=}\bigcup_{B\in\mathcal{B}}B\) e \(\displaystyle\emptyset=\bigcup_\emptyset B\), per cui \(\Set{\emptyset,X}\in\Tau_{\mathcal{B}}\).
        \item Sia ora \(\{U_j\}_{j\in J}\) una famiglia di aperti di \(X\), per costruzione avremo
              \[
                  U_j=\bigcup_{k\in K(j)}B_k,\text{ con }B_k\in\mathcal{B},
              \]
              da cui
              \[
                  \bigcup_{j\in J}U_j=\bigcup_{j\in J}\left(\bigcup_{k\in K(j)}B_k\right)=\bigcup_{\substack{k\in K(j)\\j\in J}}B_k,
              \]
              ovvero un'unione di elementi di \(\mathcal{B}\) che è pertanto un aperto.
        \item Sia infine \(U_1,\dots,U_n\) una famiglia finita di aperti, dobbiamo verificare che
              \[
                  \bigcap_{i=1}^n U_i
              \]
              sia aperto.
              Per farlo basta verificare che \(U_1\cap U_2\) sia aperto, ma ciò segue banalmente da \((2)\), infatti
              \begin{gather*}
                  U_1=\bigcup_{h\in H}B_h,\text{ con }B_h\in\mathcal{B};\\
                  U_2=\bigcup_{k\in K}B_k,\text{ con }B_k\in\mathcal{B},
              \end{gather*}
              da cui
              \[
                  U_1\cap U_2=\bigcup_{\substack{h\in H\\k\in K}}\left(B_h\cap B_k\right)\overset{(2)}{=}\bigcup_{\substack{h\in H\\k\in K}}\left(\bigcup_{j\in J}B_j\right)\in\Tau_{\mathcal{B}}.\qedhere
              \]
    \end{itemize}
\end{proof}

\begin{oss}
    Prese due basi \(\mathcal{B}\)e \(\mathcal{B}'\) dello stesso insieme \(X\), avremo che
    \[
        \Tau_{\mathcal{B}}<\Tau_{\mathcal{B}'}\iff \mathcal{B}\subset\mathcal{B}'.
    \]
\end{oss}
%%%%%%%%%%%%%%%%%%%%%%%%%%%%%%%%%%%%%%%%%
%
%LEZIONE 25/02/2016 - PRIMA SETTIMANA (3)
%
%%%%%%%%%%%%%%%%%%%%%%%%%%%%%%%%%%%%%%%%%
\begin{defn}{Intorno di un punto}{intorno}\index{Intorno}
    Preso un punto \(x\in X\) in uno spazio topologico \((X,\Tau)\), un \emph{intorno} di \(x\) è un qualsiasi aperto \(U\) che contiene \(x\)
\end{defn}

\begin{notz}
    Indicheremo con \(U_x\) un intorno di \(x\in X\).
\end{notz}

\begin{defn}{Intorno di un insieme}{intornoInsieme}
    Sia \((X,\Tau)\) uno spazio topologico e sia \(K\subset X\) un sottoinsieme qualsiasi.
    Un \emph{intorno} di \(K\) è un aperto \(U\) che contiene \(K\).
\end{defn}

\begin{prop}{Caratterizzazione degli aperti tramite intorni}{caratApertiIntorni}
    Sia \((X,\Tau)\) uno spazio topologico e sia \(A\subset X\).
    Allora \(A\) è aperto se e soltanto se
    \[
        \fa x\in A \,\ex U_x:U_x\subseteq A.
    \]
\end{prop}

\begin{proof}
    \graffito{\(\Rightarrow)\)}Se \(A\) è aperto basta prendere \(U=A\), in quanto \(A\) è intorno di ogni suo punto.\\
    \graffito{\(\Leftarrow)\)}Supponiamo che, preso \(x\in A\), \(U_x\) sia un suo intorno, è facile mostrare che
    \[
        A=\bigcup_{x\in A}U_x.
    \]
    Per cui \(A\) è aperto in quanto unione arbitraria di aperti.
\end{proof}
%%%%%%%%%%%%%
%SUCCESSIONI%
%%%%%%%%%%%%%
\section{Successioni}

In questo paragrafo faremo riferimento allo spazio topologico \(X\) come la coppia \((X,\Tau)\).

\begin{defn}{Successione}{successione}\index{Successione}
    Una \emph{successione} \(\{X_n\}_{n\in\N}\) in uno spazio topologico \(X\) è un'applicazione
    \[
        \N\to X,n\mapsto x_n.
    \]
\end{defn}

\begin{defn}{Successione convergente}{successioneConvergente}\index{Successione!convergente}
    Una successione \(\{X_n\}_{n\in\N}\) in uno spazio topologico \(X\) si dice \emph{convergente} ad \(x\in X\) se, per ogni intorno \(U\) di \(x\)
    \[
        \ex N\in\N\text{ tale che }x_n\in U,\,\fa n>N.
    \]
\end{defn}

\begin{notz}
    Quando una successione è convergente si scrive \(x_n\to x\) oppure
    \[
        \lim_{n\to +\infty} x_n=x.
    \]
\end{notz}

\begin{ese}
    Se \((X,d)\) è uno spazio metrico, ovvero uno spazio topologico indotto dalla metrica, ritroviamo facilmente la definizione che abbiamo in analisi.
    Infatti, dal momento che i dischi sono una base della topologia, è sufficiente che \(U\) sia un disco, per cui, preso \(U=D_\e(x)\), avremo
    \[
        x_n\to x\iff \fa\e>0\,\ex N\in\N:d(x_n,x)<\e\graffito{\(d(x_n,x)\iff x_n\in D_\e(x)\)},\fa n>\N.
    \]
\end{ese}

\begin{ese}
    Se \(X\) ha la topologia banale, allora ogni successione \(\N\to X\) converge ad ogni punto di \(X\).
\end{ese}

\begin{ese}
    Se \(X\) ha la topologia discreta, allora, le uniche successioni convergenti sono quelle costanti o definitivamente costanti,ovvero quelle dove
    \[
        x_n=x_{n+1},\,\fa n>N.
    \]
\end{ese}